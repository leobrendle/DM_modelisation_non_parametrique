% Options for packages loaded elsewhere
\PassOptionsToPackage{unicode}{hyperref}
\PassOptionsToPackage{hyphens}{url}
\documentclass[
]{article}
\usepackage{xcolor}
\usepackage[margin=1in]{geometry}
\usepackage{amsmath,amssymb}
\setcounter{secnumdepth}{-\maxdimen} % remove section numbering
\usepackage{iftex}
\ifPDFTeX
  \usepackage[T1]{fontenc}
  \usepackage[utf8]{inputenc}
  \usepackage{textcomp} % provide euro and other symbols
\else % if luatex or xetex
  \usepackage{unicode-math} % this also loads fontspec
  \defaultfontfeatures{Scale=MatchLowercase}
  \defaultfontfeatures[\rmfamily]{Ligatures=TeX,Scale=1}
\fi
\usepackage{lmodern}
\ifPDFTeX\else
  % xetex/luatex font selection
\fi
% Use upquote if available, for straight quotes in verbatim environments
\IfFileExists{upquote.sty}{\usepackage{upquote}}{}
\IfFileExists{microtype.sty}{% use microtype if available
  \usepackage[]{microtype}
  \UseMicrotypeSet[protrusion]{basicmath} % disable protrusion for tt fonts
}{}
\makeatletter
\@ifundefined{KOMAClassName}{% if non-KOMA class
  \IfFileExists{parskip.sty}{%
    \usepackage{parskip}
  }{% else
    \setlength{\parindent}{0pt}
    \setlength{\parskip}{6pt plus 2pt minus 1pt}}
}{% if KOMA class
  \KOMAoptions{parskip=half}}
\makeatother
\usepackage{color}
\usepackage{fancyvrb}
\newcommand{\VerbBar}{|}
\newcommand{\VERB}{\Verb[commandchars=\\\{\}]}
\DefineVerbatimEnvironment{Highlighting}{Verbatim}{commandchars=\\\{\}}
% Add ',fontsize=\small' for more characters per line
\usepackage{framed}
\definecolor{shadecolor}{RGB}{248,248,248}
\newenvironment{Shaded}{\begin{snugshade}}{\end{snugshade}}
\newcommand{\AlertTok}[1]{\textcolor[rgb]{0.94,0.16,0.16}{#1}}
\newcommand{\AnnotationTok}[1]{\textcolor[rgb]{0.56,0.35,0.01}{\textbf{\textit{#1}}}}
\newcommand{\AttributeTok}[1]{\textcolor[rgb]{0.13,0.29,0.53}{#1}}
\newcommand{\BaseNTok}[1]{\textcolor[rgb]{0.00,0.00,0.81}{#1}}
\newcommand{\BuiltInTok}[1]{#1}
\newcommand{\CharTok}[1]{\textcolor[rgb]{0.31,0.60,0.02}{#1}}
\newcommand{\CommentTok}[1]{\textcolor[rgb]{0.56,0.35,0.01}{\textit{#1}}}
\newcommand{\CommentVarTok}[1]{\textcolor[rgb]{0.56,0.35,0.01}{\textbf{\textit{#1}}}}
\newcommand{\ConstantTok}[1]{\textcolor[rgb]{0.56,0.35,0.01}{#1}}
\newcommand{\ControlFlowTok}[1]{\textcolor[rgb]{0.13,0.29,0.53}{\textbf{#1}}}
\newcommand{\DataTypeTok}[1]{\textcolor[rgb]{0.13,0.29,0.53}{#1}}
\newcommand{\DecValTok}[1]{\textcolor[rgb]{0.00,0.00,0.81}{#1}}
\newcommand{\DocumentationTok}[1]{\textcolor[rgb]{0.56,0.35,0.01}{\textbf{\textit{#1}}}}
\newcommand{\ErrorTok}[1]{\textcolor[rgb]{0.64,0.00,0.00}{\textbf{#1}}}
\newcommand{\ExtensionTok}[1]{#1}
\newcommand{\FloatTok}[1]{\textcolor[rgb]{0.00,0.00,0.81}{#1}}
\newcommand{\FunctionTok}[1]{\textcolor[rgb]{0.13,0.29,0.53}{\textbf{#1}}}
\newcommand{\ImportTok}[1]{#1}
\newcommand{\InformationTok}[1]{\textcolor[rgb]{0.56,0.35,0.01}{\textbf{\textit{#1}}}}
\newcommand{\KeywordTok}[1]{\textcolor[rgb]{0.13,0.29,0.53}{\textbf{#1}}}
\newcommand{\NormalTok}[1]{#1}
\newcommand{\OperatorTok}[1]{\textcolor[rgb]{0.81,0.36,0.00}{\textbf{#1}}}
\newcommand{\OtherTok}[1]{\textcolor[rgb]{0.56,0.35,0.01}{#1}}
\newcommand{\PreprocessorTok}[1]{\textcolor[rgb]{0.56,0.35,0.01}{\textit{#1}}}
\newcommand{\RegionMarkerTok}[1]{#1}
\newcommand{\SpecialCharTok}[1]{\textcolor[rgb]{0.81,0.36,0.00}{\textbf{#1}}}
\newcommand{\SpecialStringTok}[1]{\textcolor[rgb]{0.31,0.60,0.02}{#1}}
\newcommand{\StringTok}[1]{\textcolor[rgb]{0.31,0.60,0.02}{#1}}
\newcommand{\VariableTok}[1]{\textcolor[rgb]{0.00,0.00,0.00}{#1}}
\newcommand{\VerbatimStringTok}[1]{\textcolor[rgb]{0.31,0.60,0.02}{#1}}
\newcommand{\WarningTok}[1]{\textcolor[rgb]{0.56,0.35,0.01}{\textbf{\textit{#1}}}}
\usepackage{graphicx}
\makeatletter
\newsavebox\pandoc@box
\newcommand*\pandocbounded[1]{% scales image to fit in text height/width
  \sbox\pandoc@box{#1}%
  \Gscale@div\@tempa{\textheight}{\dimexpr\ht\pandoc@box+\dp\pandoc@box\relax}%
  \Gscale@div\@tempb{\linewidth}{\wd\pandoc@box}%
  \ifdim\@tempb\p@<\@tempa\p@\let\@tempa\@tempb\fi% select the smaller of both
  \ifdim\@tempa\p@<\p@\scalebox{\@tempa}{\usebox\pandoc@box}%
  \else\usebox{\pandoc@box}%
  \fi%
}
% Set default figure placement to htbp
\def\fps@figure{htbp}
\makeatother
\setlength{\emergencystretch}{3em} % prevent overfull lines
\providecommand{\tightlist}{%
  \setlength{\itemsep}{0pt}\setlength{\parskip}{0pt}}
\usepackage{bookmark}
\IfFileExists{xurl.sty}{\usepackage{xurl}}{} % add URL line breaks if available
\urlstyle{same}
\hypersetup{
  pdftitle={TP Apprentissage statistique : modelisation non-paramétrique par des fonctions B-splines},
  pdfauthor={Fouad AFANE, Emilie GALLAND, Léo BRENDLE},
  hidelinks,
  pdfcreator={LaTeX via pandoc}}

\title{TP Apprentissage statistique : modelisation non-paramétrique par
des fonctions B-splines}
\author{Fouad AFANE, Emilie GALLAND, Léo BRENDLE}
\date{Décembre 2025}

\begin{document}
\maketitle

\subsection{Introduction}\label{introduction}

L'objectif de ce TP est de modéliser le prix des appartements à Varsovie
en utilisant des méthodes non paramétriques.

\begin{Shaded}
\begin{Highlighting}[]
\FunctionTok{require}\NormalTok{(ggplot2)}
\FunctionTok{require}\NormalTok{(HRW)}
\FunctionTok{require}\NormalTok{(mgcv)}
\FunctionTok{require}\NormalTok{(splines)}
\FunctionTok{require}\NormalTok{(dplyr)}
\FunctionTok{data}\NormalTok{(WarsawApts)}
\FunctionTok{str}\NormalTok{(WarsawApts)}
\end{Highlighting}
\end{Shaded}

\begin{verbatim}
## 'data.frame':    409 obs. of  6 variables:
##  $ surface          : num  20 27 28 28 30 32 32 36 36 37 ...
##  $ district         : chr  "Srodmiescie" "Wola" "Mokotow" "Mokotow" ...
##  $ n.rooms          : int  1 1 1 2 1 2 1 2 2 2 ...
##  $ floor            : int  7 1 1 4 1 3 7 6 5 7 ...
##  $ construction.date: int  1970 1962 1950 1968 1952 2007 1961 1965 1968 1972 ...
##  $ areaPerMzloty    : num  95.2 117.4 114.3 114.3 93.8 ...
\end{verbatim}

Les variables du jeu de données sont les suivantes :

\begin{itemize}
\tightlist
\item
  \textbf{surface} : surface en mètres carrés\\
\item
  \textbf{district} : district (variable facteur à 4 modalités)\\
\item
  \textbf{n.rooms} : nombre de pièces\\
\item
  \textbf{floor} : étage\\
\item
  \textbf{construction.date} : année de construction\\
\item
  \textbf{areaPerMzloty} : prix au mètre carré (variable à prédire)
\end{itemize}

\subsection{Partie 1 : Prédiction avec une seule variable
explicative}\label{partie-1-pruxe9diction-avec-une-seule-variable-explicative}

\subsubsection{1. Exploration des
variables}\label{exploration-des-variables}

Pour commencer, on veut modéliser le prix des appartements en fonction
d'une seule variable. Pour choisir la plus adaptée, traçons le prix des
appartements en fonction des différentes variables. On considère que
l'on dispose de 4 variables explicatives quantitatives et une variable
qualitative (district), que l'on va écarter de l'étude pour le moment.

\begin{Shaded}
\begin{Highlighting}[]
\FunctionTok{library}\NormalTok{(ggplot2)}
\FunctionTok{library}\NormalTok{(patchwork)}
\end{Highlighting}
\end{Shaded}

\begin{verbatim}
## Warning: le package 'patchwork' a été compilé avec la version R 4.4.3
\end{verbatim}

\begin{Shaded}
\begin{Highlighting}[]
\NormalTok{p1 }\OtherTok{\textless{}{-}} \FunctionTok{ggplot}\NormalTok{(WarsawApts, }\FunctionTok{aes}\NormalTok{(}\AttributeTok{x =}\NormalTok{ surface, }\AttributeTok{y =}\NormalTok{ areaPerMzloty)) }\SpecialCharTok{+}
  \FunctionTok{geom\_point}\NormalTok{(}\AttributeTok{color =} \StringTok{"red"}\NormalTok{) }\SpecialCharTok{+}
  \FunctionTok{labs}\NormalTok{(}\AttributeTok{title =} \StringTok{"Prix en fonction de surface"}\NormalTok{,}
       \AttributeTok{x =} \StringTok{"Surface (m²)"}\NormalTok{, }\AttributeTok{y =} \StringTok{"Prix au m²"}\NormalTok{)}

\NormalTok{p2 }\OtherTok{\textless{}{-}} \FunctionTok{ggplot}\NormalTok{(WarsawApts, }\FunctionTok{aes}\NormalTok{(}\AttributeTok{x =}\NormalTok{ n.rooms, }\AttributeTok{y =}\NormalTok{ areaPerMzloty)) }\SpecialCharTok{+}
  \FunctionTok{geom\_point}\NormalTok{(}\AttributeTok{color =} \StringTok{"blue"}\NormalTok{) }\SpecialCharTok{+}
  \FunctionTok{labs}\NormalTok{(}\AttributeTok{title =} \StringTok{"Prix en fonction du nb de chambres"}\NormalTok{,}
       \AttributeTok{x =} \StringTok{"Nombre de chambres"}\NormalTok{, }\AttributeTok{y =} \StringTok{"Prix au m²"}\NormalTok{)}

\NormalTok{p3 }\OtherTok{\textless{}{-}} \FunctionTok{ggplot}\NormalTok{(WarsawApts, }\FunctionTok{aes}\NormalTok{(}\AttributeTok{x =}\NormalTok{ floor, }\AttributeTok{y =}\NormalTok{ areaPerMzloty)) }\SpecialCharTok{+}
  \FunctionTok{geom\_point}\NormalTok{(}\AttributeTok{color =} \StringTok{"darkgreen"}\NormalTok{) }\SpecialCharTok{+}
  \FunctionTok{labs}\NormalTok{(}\AttributeTok{title =} \StringTok{"Prix en fonction de l\textquotesingle{}étage"}\NormalTok{,}
       \AttributeTok{x =} \StringTok{"Étage"}\NormalTok{, }\AttributeTok{y =} \StringTok{"Prix au m²"}\NormalTok{)}

\NormalTok{p4 }\OtherTok{\textless{}{-}} \FunctionTok{ggplot}\NormalTok{(WarsawApts, }\FunctionTok{aes}\NormalTok{(}\AttributeTok{x =}\NormalTok{ construction.date, }\AttributeTok{y =}\NormalTok{ areaPerMzloty)) }\SpecialCharTok{+}
  \FunctionTok{geom\_point}\NormalTok{(}\AttributeTok{color =} \StringTok{"purple"}\NormalTok{) }\SpecialCharTok{+}
  \FunctionTok{labs}\NormalTok{(}\AttributeTok{title =} \StringTok{"Prix en fonction de l\textquotesingle{}année de construction"}\NormalTok{,}
       \AttributeTok{x =} \StringTok{"Année de construction"}\NormalTok{, }\AttributeTok{y =} \StringTok{"Prix au m²"}\NormalTok{)}

\NormalTok{(p1 }\SpecialCharTok{|}\NormalTok{ p2) }\SpecialCharTok{/}
\NormalTok{(p3 }\SpecialCharTok{|}\NormalTok{ p4)}
\end{Highlighting}
\end{Shaded}

\pandocbounded{\includegraphics[keepaspectratio]{DM_modelisation_non_param_files/figure-latex/unnamed-chunk-2-1.pdf}}

\subsubsection{Choix d'une variable explicative pour la modélisation
univariée}\label{choix-dune-variable-explicative-pour-la-moduxe9lisation-univariuxe9e}

Les graphiques précédents (P1--P4) représentent le prix au m² en
fonction de la surface, de l'étage et de l'année de construction.
Visuellement, aucune de ces relations ne semble strictement linéaire, en
particulier pour \texttt{construction.date} où l'on observe une forme en
« bosse » : les immeubles d'une période intermédiaire paraissent plus
chers que les constructions très anciennes ou très récentes.

Afin de choisir une variable explicative unique pour la suite, nous
avons comparé, pour chacune des variables quantitatives
\texttt{surface}, \texttt{floor} et \texttt{construction.date}, trois
modèles paramétriques : un modèle linéaire simple, un polynôme d'ordre 2
et un polynôme d'ordre 3 de \texttt{areaPerMzloty} en fonction de cette
variable. Pour chaque combinaison, nous avons relevé le coefficient de
détermination R² et le critère d'information AIC, présentés dans le
tableau ci-dessous.

\begin{Shaded}
\begin{Highlighting}[]
\CommentTok{\# On suppose que les données sont dans \textasciigrave{}donnees\textasciigrave{}}
\CommentTok{\# et que la variable réponse s\textquotesingle{}appelle \textasciigrave{}areaPerMzloty\textasciigrave{}.}

\NormalTok{donnees }\OtherTok{\textless{}{-}}\NormalTok{ WarsawApts}
\NormalTok{vars }\OtherTok{\textless{}{-}} \FunctionTok{c}\NormalTok{(}\StringTok{"surface"}\NormalTok{, }\StringTok{"floor"}\NormalTok{, }\StringTok{"construction.date"}\NormalTok{)}

\NormalTok{res }\OtherTok{\textless{}{-}} \FunctionTok{data.frame}\NormalTok{(}
  \AttributeTok{var       =}\NormalTok{ vars,}
  \AttributeTok{R2\_lin    =} \ConstantTok{NA}\NormalTok{,}
  \AttributeTok{R2\_poly2  =} \ConstantTok{NA}\NormalTok{,}
  \AttributeTok{R2\_poly3  =} \ConstantTok{NA}\NormalTok{,}
  \AttributeTok{AIC\_lin   =} \ConstantTok{NA}\NormalTok{,}
  \AttributeTok{AIC\_poly2 =} \ConstantTok{NA}\NormalTok{,}
  \AttributeTok{AIC\_poly3 =} \ConstantTok{NA}
\NormalTok{)}

\ControlFlowTok{for}\NormalTok{ (j }\ControlFlowTok{in} \FunctionTok{seq\_along}\NormalTok{(vars)) \{}
\NormalTok{  xname }\OtherTok{\textless{}{-}}\NormalTok{ vars[j]}
\NormalTok{  x     }\OtherTok{\textless{}{-}}\NormalTok{ donnees[[xname]]           }\CommentTok{\# on extrait la colonne}
  
  \CommentTok{\# Modèle linéaire simple}
\NormalTok{  m\_lin }\OtherTok{\textless{}{-}} \FunctionTok{lm}\NormalTok{(donnees}\SpecialCharTok{$}\NormalTok{areaPerMzloty }\SpecialCharTok{\textasciitilde{}}\NormalTok{ x)}
  
  \CommentTok{\# Polynôme ordre 2}
\NormalTok{  m\_poly2 }\OtherTok{\textless{}{-}} \FunctionTok{lm}\NormalTok{(donnees}\SpecialCharTok{$}\NormalTok{areaPerMzloty }\SpecialCharTok{\textasciitilde{}} \FunctionTok{poly}\NormalTok{(x, }\DecValTok{2}\NormalTok{))}
  
  \CommentTok{\# Polynôme ordre 3}
\NormalTok{  m\_poly3 }\OtherTok{\textless{}{-}} \FunctionTok{lm}\NormalTok{(donnees}\SpecialCharTok{$}\NormalTok{areaPerMzloty }\SpecialCharTok{\textasciitilde{}} \FunctionTok{poly}\NormalTok{(x, }\DecValTok{3}\NormalTok{))}
  
\NormalTok{  res}\SpecialCharTok{$}\NormalTok{R2\_lin[j]    }\OtherTok{\textless{}{-}} \FunctionTok{summary}\NormalTok{(m\_lin)}\SpecialCharTok{$}\NormalTok{r.squared}
\NormalTok{  res}\SpecialCharTok{$}\NormalTok{R2\_poly2[j]  }\OtherTok{\textless{}{-}} \FunctionTok{summary}\NormalTok{(m\_poly2)}\SpecialCharTok{$}\NormalTok{r.squared}
\NormalTok{  res}\SpecialCharTok{$}\NormalTok{R2\_poly3[j]  }\OtherTok{\textless{}{-}} \FunctionTok{summary}\NormalTok{(m\_poly3)}\SpecialCharTok{$}\NormalTok{r.squared}
  
\NormalTok{  res}\SpecialCharTok{$}\NormalTok{AIC\_lin[j]   }\OtherTok{\textless{}{-}} \FunctionTok{AIC}\NormalTok{(m\_lin)}
\NormalTok{  res}\SpecialCharTok{$}\NormalTok{AIC\_poly2[j] }\OtherTok{\textless{}{-}} \FunctionTok{AIC}\NormalTok{(m\_poly2)}
\NormalTok{  res}\SpecialCharTok{$}\NormalTok{AIC\_poly3[j] }\OtherTok{\textless{}{-}} \FunctionTok{AIC}\NormalTok{(m\_poly3)}
\NormalTok{\}}

\NormalTok{res}
\end{Highlighting}
\end{Shaded}

\begin{verbatim}
##                 var     R2_lin   R2_poly2   R2_poly3  AIC_lin AIC_poly2
## 1           surface 0.05481273 0.06277235 0.10011316 3675.828  3674.370
## 2             floor 0.01314080 0.02697317 0.03051805 3693.474  3689.701
## 3 construction.date 0.01262920 0.23920797 0.29819879 3693.686  3589.066
##   AIC_poly3
## 1  3659.741
## 2  3690.208
## 3  3558.056
\end{verbatim}

Les résultats montrent que, pour \texttt{surface} et \texttt{floor},
même un polynôme cubique n'explique qu'une faible part de la variabilité
du prix au m² (R² inférieur à 0,10) et conduit à des AIC relativement
élevés (autour de 3 660--3 690). En revanche, pour
\texttt{construction.date}, le polynôme d'ordre 3 atteint un R² proche
de 0,30 et un AIC nettement plus faible (environ 3 558), ce qui en fait
de loin le meilleur modèle paramétrique parmi ceux testés.

Nous retenons donc \textbf{l'année de construction
(\texttt{construction.date})} comme variable explicative unique pour la
première étape de la modélisation. Dans la suite, nous comparerons ce
modèle paramétrique cubique à un modèle non paramétrique à base de
splines, mieux adapté à la forme non linéaire observée entre l'année de
construction et le prix au m².

\subsubsection{2. Modélisation paramétrique du prix en fonction de
l'année
(Fouad)}\label{moduxe9lisation-paramuxe9trique-du-prix-en-fonction-de-lannuxe9e-fouad}

\begin{Shaded}
\begin{Highlighting}[]
\FunctionTok{library}\NormalTok{(HRW)}
\FunctionTok{data}\NormalTok{(}\StringTok{"WarsawApts"}\NormalTok{)      }\CommentTok{\# recharge le jeu de données tel qu\textquotesingle{}il est dans le package}
\NormalTok{donnees }\OtherTok{\textless{}{-}}\NormalTok{ WarsawApts   }\CommentTok{\# on travaille avec \textasciigrave{}donnees\textasciigrave{} pour la suite}

\FunctionTok{library}\NormalTok{(ggplot2)}

\NormalTok{m\_year\_poly2 }\OtherTok{\textless{}{-}} \FunctionTok{lm}\NormalTok{(areaPerMzloty }\SpecialCharTok{\textasciitilde{}} \FunctionTok{poly}\NormalTok{(construction.date, }\DecValTok{2}\NormalTok{), }\AttributeTok{data =}\NormalTok{ donnees)}
\NormalTok{m\_year\_poly3 }\OtherTok{\textless{}{-}} \FunctionTok{lm}\NormalTok{(areaPerMzloty }\SpecialCharTok{\textasciitilde{}} \FunctionTok{poly}\NormalTok{(construction.date, }\DecValTok{3}\NormalTok{), }\AttributeTok{data =}\NormalTok{ donnees)}

\NormalTok{newdat }\OtherTok{\textless{}{-}} \FunctionTok{data.frame}\NormalTok{(}
  \AttributeTok{construction.date =} \FunctionTok{seq}\NormalTok{(}\FunctionTok{min}\NormalTok{(donnees}\SpecialCharTok{$}\NormalTok{construction.date),}
                          \FunctionTok{max}\NormalTok{(donnees}\SpecialCharTok{$}\NormalTok{construction.date),}
                          \AttributeTok{length.out =} \DecValTok{200}\NormalTok{)}
\NormalTok{)}
\NormalTok{newdat}\SpecialCharTok{$}\NormalTok{pred2 }\OtherTok{\textless{}{-}} \FunctionTok{predict}\NormalTok{(m\_year\_poly2, newdat)}
\NormalTok{newdat}\SpecialCharTok{$}\NormalTok{pred3 }\OtherTok{\textless{}{-}} \FunctionTok{predict}\NormalTok{(m\_year\_poly3, newdat)}

\NormalTok{newdat\_long }\OtherTok{\textless{}{-}} \FunctionTok{rbind}\NormalTok{(}
  \FunctionTok{data.frame}\NormalTok{(}
    \AttributeTok{construction.date =}\NormalTok{ newdat}\SpecialCharTok{$}\NormalTok{construction.date,}
    \AttributeTok{prediction        =}\NormalTok{ newdat}\SpecialCharTok{$}\NormalTok{pred2,}
    \AttributeTok{modele            =} \StringTok{"Polynôme degré 2"}
\NormalTok{  ),}
  \FunctionTok{data.frame}\NormalTok{(}
    \AttributeTok{construction.date =}\NormalTok{ newdat}\SpecialCharTok{$}\NormalTok{construction.date,}
    \AttributeTok{prediction        =}\NormalTok{ newdat}\SpecialCharTok{$}\NormalTok{pred3,}
    \AttributeTok{modele            =} \StringTok{"Polynôme degré 3"}
\NormalTok{  )}
\NormalTok{)}

\FunctionTok{ggplot}\NormalTok{(donnees, }\FunctionTok{aes}\NormalTok{(}\AttributeTok{x =}\NormalTok{ construction.date, }\AttributeTok{y =}\NormalTok{ areaPerMzloty)) }\SpecialCharTok{+}
  \FunctionTok{geom\_point}\NormalTok{(}\AttributeTok{alpha =} \FloatTok{0.5}\NormalTok{, }\AttributeTok{size =} \DecValTok{1}\NormalTok{) }\SpecialCharTok{+}
  \FunctionTok{geom\_line}\NormalTok{(}\AttributeTok{data =}\NormalTok{ newdat\_long,}
            \FunctionTok{aes}\NormalTok{(}\AttributeTok{y =}\NormalTok{ prediction, }\AttributeTok{colour =}\NormalTok{ modele),}
            \AttributeTok{linewidth =} \FloatTok{1.1}\NormalTok{) }\SpecialCharTok{+}
  \FunctionTok{scale\_colour\_manual}\NormalTok{(}\AttributeTok{values =} \FunctionTok{c}\NormalTok{(}\StringTok{"Polynôme degré 2"} \OtherTok{=} \StringTok{"blue"}\NormalTok{,}
                                 \StringTok{"Polynôme degré 3"} \OtherTok{=} \StringTok{"red"}\NormalTok{)) }\SpecialCharTok{+}
  \FunctionTok{labs}\NormalTok{(}
    \AttributeTok{x =} \StringTok{"Année de construction"}\NormalTok{,}
    \AttributeTok{y =} \StringTok{"Prix au m²"}\NormalTok{,}
    \AttributeTok{title =} \StringTok{"Prix au m² \textasciitilde{} année : polynômes d\textquotesingle{}ordre 2 et 3"}\NormalTok{,}
    \AttributeTok{colour =} \StringTok{""}
\NormalTok{  ) }\SpecialCharTok{+}
  \FunctionTok{theme\_bw}\NormalTok{()}
\end{Highlighting}
\end{Shaded}

\pandocbounded{\includegraphics[keepaspectratio]{DM_modelisation_non_param_files/figure-latex/unnamed-chunk-3-1.pdf}}
\#\#\# Résultat des modèles paramétriques polynomiaux en fonction de
l'année de construction

La figure ci-dessus représente le prix au m² en fonction de l'année de
construction, ainsi que les courbes ajustées par un polynôme d'ordre 2
et un polynôme d'ordre 3. Le modèle quadratique reproduit grossièrement
la forme en cloche observée dans les données, mais reste assez rigide :
la bosse centrale est sous-estimée et la baisse pour les immeubles
récents n'est pas parfaitement capturée.

Le polynôme d'ordre 3 offre une meilleure flexibilité. La courbe rouge
suit plus fidèlement la hausse des prix au m² pour les immeubles
construits autour de la période intermédiaire, puis la baisse pour les
constructions les plus récentes. Cela se traduit par un R² plus élevé
(environ 0,30 contre 0,24 pour l'ordre 2) et un AIC plus faible. Ce
modèle cubique constitue donc notre \textbf{meilleur modèle paramétrique
univarié} pour l'effet de l'année de construction.

Cependant, il reste un modèle \textbf{global} : une seule fonction
cubique doit décrire l'ensemble de la période 1930--2008, ce qui limite
la capacité à s'adapter \textbf{localement} aux différentes époques.
Cette limitation motive l'introduction, dans la section suivante, d'un
modèle non paramétrique à base de splines pénalisées.

\subsubsection{3. Modélisation non paramétrique du prix en fonction de
l'année
(Emilie)}\label{moduxe9lisation-non-paramuxe9trique-du-prix-en-fonction-de-lannuxe9e-emilie}

On commence par normaliser la variable année :

\begin{Shaded}
\begin{Highlighting}[]
\NormalTok{annee }\OtherTok{\textless{}{-}}\NormalTok{ WarsawApts}\SpecialCharTok{$}\NormalTok{construction.date}
\NormalTok{WarsawApts}\SpecialCharTok{$}\NormalTok{construction.date }\OtherTok{\textless{}{-}}\NormalTok{ (annee}\SpecialCharTok{{-}}\FunctionTok{min}\NormalTok{(annee))}\SpecialCharTok{/}\NormalTok{(}\FunctionTok{max}\NormalTok{(annee)}\SpecialCharTok{{-}}\FunctionTok{min}\NormalTok{(annee)) }

\NormalTok{WarsawApts }\OtherTok{\textless{}{-}}\NormalTok{ WarsawApts }\SpecialCharTok{\%\textgreater{}\%}
  \FunctionTok{arrange}\NormalTok{(construction.date)}
\end{Highlighting}
\end{Shaded}

Ensuite on définit les modèles GAM avec la paramètre de lissage sp =
\lambda que l'on cherche par validation croisée par blocs.

\begin{Shaded}
\begin{Highlighting}[]
\NormalTok{B }\OtherTok{\textless{}{-}} \DecValTok{5}
\NormalTok{df }\OtherTok{\textless{}{-}}\NormalTok{ WarsawApts[}\FunctionTok{sample}\NormalTok{(}\FunctionTok{nrow}\NormalTok{(WarsawApts)),]}

\NormalTok{n }\OtherTok{\textless{}{-}} \FunctionTok{nrow}\NormalTok{(df) }
\NormalTok{block\_size }\OtherTok{\textless{}{-}} \FunctionTok{floor}\NormalTok{(n }\SpecialCharTok{/}\NormalTok{ B)}

\NormalTok{grid }\OtherTok{\textless{}{-}} \DecValTok{10}\SpecialCharTok{\^{}}\FunctionTok{seq}\NormalTok{(}\DecValTok{2}\NormalTok{, }\SpecialCharTok{{-}}\DecValTok{2}\NormalTok{, }\AttributeTok{length =} \DecValTok{100}\NormalTok{)}
\NormalTok{vec\_mse }\OtherTok{\textless{}{-}} \FunctionTok{numeric}\NormalTok{(}\FunctionTok{length}\NormalTok{(grid))}

\ControlFlowTok{for}\NormalTok{ (i }\ControlFlowTok{in} \DecValTok{1}\SpecialCharTok{:}\FunctionTok{length}\NormalTok{(grid)) \{}

\NormalTok{  lambda }\OtherTok{\textless{}{-}}\NormalTok{ grid[i]}
\NormalTok{  mse\_blocks }\OtherTok{\textless{}{-}} \FunctionTok{numeric}\NormalTok{(B)}

  \ControlFlowTok{for}\NormalTok{ (b }\ControlFlowTok{in} \DecValTok{1}\SpecialCharTok{:}\NormalTok{B) \{}
\NormalTok{    test\_idx }\OtherTok{\textless{}{-}}\NormalTok{ ((b }\SpecialCharTok{{-}} \DecValTok{1}\NormalTok{) }\SpecialCharTok{*}\NormalTok{ block\_size }\SpecialCharTok{+} \DecValTok{1}\NormalTok{) }\SpecialCharTok{:}\NormalTok{ (b }\SpecialCharTok{*}\NormalTok{ block\_size)}
    \ControlFlowTok{if}\NormalTok{ (b }\SpecialCharTok{==}\NormalTok{ B)\{}
\NormalTok{      test\_idx }\OtherTok{\textless{}{-}}\NormalTok{ ((b }\SpecialCharTok{{-}} \DecValTok{1}\NormalTok{) }\SpecialCharTok{*}\NormalTok{ block\_size }\SpecialCharTok{+} \DecValTok{1}\NormalTok{) }\SpecialCharTok{:}\NormalTok{ n}
\NormalTok{    \} }
\NormalTok{    train\_data }\OtherTok{\textless{}{-}}\NormalTok{ df[}\SpecialCharTok{{-}}\NormalTok{test\_idx, ] }\SpecialCharTok{\%\textgreater{}\%}
      \FunctionTok{arrange}\NormalTok{(construction.date)}
\NormalTok{    test\_data  }\OtherTok{\textless{}{-}}\NormalTok{ df[test\_idx, ]}\SpecialCharTok{\%\textgreater{}\%}
      \FunctionTok{arrange}\NormalTok{(construction.date)}

\NormalTok{    res.fit }\OtherTok{\textless{}{-}} \FunctionTok{gam}\NormalTok{(areaPerMzloty }\SpecialCharTok{\textasciitilde{}} \FunctionTok{s}\NormalTok{(construction.date, }\AttributeTok{bs =} \StringTok{"bs"}\NormalTok{, }\AttributeTok{k =} \DecValTok{5}\NormalTok{, }\AttributeTok{m =} \FunctionTok{c}\NormalTok{(}\DecValTok{2}\NormalTok{, }\DecValTok{2}\NormalTok{)),}
      \AttributeTok{data =}\NormalTok{ train\_data, }\AttributeTok{sp =}\NormalTok{ lambda)}
    
\NormalTok{    pred }\OtherTok{\textless{}{-}} \FunctionTok{predict}\NormalTok{(res.fit, }\AttributeTok{newdata =}\NormalTok{ test\_data)}
\NormalTok{    mse\_blocks[b] }\OtherTok{\textless{}{-}} \FunctionTok{mean}\NormalTok{((test\_data}\SpecialCharTok{$}\NormalTok{areaPerMzloty }\SpecialCharTok{{-}}\NormalTok{ pred)}\SpecialCharTok{\^{}}\DecValTok{2}\NormalTok{)}
\NormalTok{  \}}
\NormalTok{  vec\_mse[i] }\OtherTok{\textless{}{-}} \FunctionTok{mean}\NormalTok{(mse\_blocks)}
\NormalTok{\}}

\CommentTok{\# plot validation croisée}

\NormalTok{mse\_grid }\OtherTok{\textless{}{-}} \FunctionTok{data.frame}\NormalTok{(grid, vec\_mse )}
\FunctionTok{ggplot}\NormalTok{(}\AttributeTok{data =}\NormalTok{ mse\_grid, }\FunctionTok{aes}\NormalTok{(}\FunctionTok{log}\NormalTok{(grid), vec\_mse))}\SpecialCharTok{+}
  \FunctionTok{geom\_point}\NormalTok{()}\SpecialCharTok{+}
  \FunctionTok{geom\_vline}\NormalTok{(}\AttributeTok{xintercept =} \FunctionTok{log}\NormalTok{(grid[}\FunctionTok{which.min}\NormalTok{(vec\_mse)]), }\AttributeTok{col =} \StringTok{"red"}\NormalTok{)}\SpecialCharTok{+}
  \FunctionTok{ggtitle}\NormalTok{(}\StringTok{"Validation croisée B{-}splines"}\NormalTok{)}\SpecialCharTok{+}
  \FunctionTok{theme\_bw}\NormalTok{()}
\end{Highlighting}
\end{Shaded}

\pandocbounded{\includegraphics[keepaspectratio]{DM_modelisation_non_param_files/figure-latex/unnamed-chunk-5-1.pdf}}

\begin{Shaded}
\begin{Highlighting}[]
\NormalTok{lambda\_cv }\OtherTok{\textless{}{-}}\NormalTok{ grid[}\FunctionTok{which.min}\NormalTok{(vec\_mse)] }
\NormalTok{lambda\_cv}
\end{Highlighting}
\end{Shaded}

\begin{verbatim}
## [1] 0.7924829
\end{verbatim}

On choisit alors le lambda qui minimise le mse. On obtient alors le
modèle suivant :

\begin{Shaded}
\begin{Highlighting}[]
\FunctionTok{attach}\NormalTok{(WarsawApts)}
\NormalTok{best.fit }\OtherTok{\textless{}{-}} \FunctionTok{gam}\NormalTok{(areaPerMzloty }\SpecialCharTok{\textasciitilde{}} \FunctionTok{s}\NormalTok{(construction.date, }\AttributeTok{bs =} \StringTok{"bs"}\NormalTok{, }\AttributeTok{k =} \DecValTok{5}\NormalTok{, }\AttributeTok{m =} \FunctionTok{c}\NormalTok{(}\DecValTok{2}\NormalTok{, }\DecValTok{2}\NormalTok{)), }\AttributeTok{data =}\NormalTok{ WarsawApts, }\AttributeTok{sp =}\NormalTok{ lambda\_cv)}
\FunctionTok{plot}\NormalTok{(construction.date, areaPerMzloty)}
\FunctionTok{lines}\NormalTok{(construction.date, }\FunctionTok{predict}\NormalTok{(best.fit), }\AttributeTok{type =} \StringTok{"l"}\NormalTok{, }\AttributeTok{col =} \StringTok{"red"}\NormalTok{)}
\end{Highlighting}
\end{Shaded}

\pandocbounded{\includegraphics[keepaspectratio]{DM_modelisation_non_param_files/figure-latex/unnamed-chunk-6-1.pdf}}

\subsection{Partie 2 : Amélioration du modèle non paramétrique en
utilisant plusieurs variables
explicatives}\label{partie-2-amuxe9lioration-du-moduxe8le-non-paramuxe9trique-en-utilisant-plusieurs-variables-explicatives}

Dans cette partie, on cherche à améliorer la prédiction du prix au m² en
utilisant plusieurs variables explicatives en même temps. L'idée est de
comparer différents types de modèles : - le modèle linéaire multiple -
le modèle additif généralisé (GAM) avec B-splines - modèle hybride, où
certaines variables sont linéaires et d'autres modélisées par des
splines

Nous utiliserons systématiquement la fonction anova() et les critères
AIC pour comparer les modèles entre eux.

\subsubsection{1. Modèle linéaire linéaire
multiple}\label{moduxe8le-linuxe9aire-linuxe9aire-multiple}

Ce modèle sert d'étalon pour comparer les gains des méthodes
non-paramétriques.

\subsubsection{2. GAM avec splines}\label{gam-avec-splines}

\subsubsection{3. Comparaison des modèles avec ANOVA et par le critère
AIC}\label{comparaison-des-moduxe8les-avec-anova-et-par-le-crituxe8re-aic}

\subsection{Partie 3 : Intégration de la variable
qualitative}\label{partie-3-intuxe9gration-de-la-variable-qualitative}

\subsection{Conclusion}\label{conclusion}

\end{document}
